%{{{Header
\documentclass[a4paper,11pt]{article}
\usepackage{anysize}
\marginsize{2cm}{2cm}{1cm}{1cm}
%\textwidth 6.0in \textheight = 664pt
\usepackage{xltxtra}
\usepackage{mathtools}
\usepackage{amsmath}
\usepackage{graphicx}
\usepackage{color}
\usepackage{xgreek}
\usepackage{framed}
\usepackage{float}
\usepackage{tkz-graph}
\usepackage{subfig}
\usepackage{fancyvrb}
\usepackage{minted}
\usepackage{listings}
\usepackage{hyperref}
\usepackage[noend]{algpseudocode}
\usepackage{algorithm}
\usepackage{enumitem}
\usepackage{relsize}
\setmainfont[Mapping=TeX-text]{DejaVu Serif}
\newcommand{\tab}{\hspace*{3em}}

%\setmainfont{Arial}
%\setsansfont{FreeSans}
%\setmonofont{FreeMono}
\begin{document}
%\def\thesubsection {\alph{subsection}}
\renewcommand{\labelenumi}{\roman{enumi})}
\renewcommand{\labelenumii}{ (\arabic{enumii}) }

\begin{titlepage}
\begin{center}
\begin{figure}[t]
     \includegraphics[scale=0.7]{title/ntua_logo}
\end{figure}
\begin{LARGE}\textbf{ΕΘΝΙΚΟ ΜΕΤΣΟΒΙΟ ΠΟΛΥΤΕΧΝΕΙΟ\\}\end{LARGE}
\vspace{5cm}
\begin{Large}
ΣΧΟΛΗ ΗΜ\&ΜΥ\\
Αλγόριθμοι και Πολυπλοκότητα\\
1\textsuperscript{η} Σειρά Γραπτών Ασκήσεων\\
Ακ. έτος 2010-2011\\
\end{Large}
\vspace{18cm}
\begin{tabular}{l r}

\begin{flushright}
\Large{Γερακάρης Βασίλης}\\
\large{Α.Μ.: 03108092}\\
\end{flushright}
\end{tabular}\\
\vspace{1cm}

\vfill
\large\today\\
\end{center}
\end{titlepage}


%}}}

%{{{ 1) Παιχνίδι Επιλογής Ακμών σε Κατευθυνόμενο Ακυκλικό Γράφημα
\section{Παιχνίδι Επιλογής Ακμών σε Κατευθυνόμενο Ακυκλικό Γράφημα} \setcounter{section}{1}
FILL ME 1!

Η υπολογιστική πολυπλοκότητα της μεθόδου μας είναι $\Theta(k + n + n) =
O(k + n)$

%}}}

%{{{ 2) Σχεδιασμός Ταξιδιού (DPV 4.13)
\section{Σχεδιασμός Ταξιδιού (DPV 4.13)}
FILL ME 2!

Η πολυπλοκότητα του αλγορίθμου είναι $\Theta(m + n)$, αφού υλοποιεί ένα BFS.

%}}}

%{{{ 3) Διαχωρισμός Γραφήματος
\section{Διαχωρισμός Γραφήματος}
FILL ME 3!

%}}}

%{{{ 4) Παιχνίδια Εξουσίας
\section{Παιχνίδια εξουσίας}
FILL ME 4!

%}}}

%{{{ 5) Αναγωγές και NP-Πληρότητα
\section{Αναγωγές και NP-Πληρότητα}
\subsection{Dense subgraph}
Dense fill!
\subsection{Μακρύ μονοπάτι}
Μακρύ indeed!
\subsection{Feedbac Vertex Set - Directed}
directed!
\subsection{Feedbac Vertex Set - Undirected}
undirected!

%}}}

\end{document}
