%{{{Header
\documentclass[a4paper,11pt]{article}
\usepackage{anysize}
\marginsize{2cm}{2cm}{1cm}{1cm}
%\textwidth 6.0in \textheight = 664pt
\usepackage{xltxtra}
\usepackage{graphicx}
\usepackage{color}
\usepackage{xgreek}
\usepackage{fancyvrb}
\usepackage{minted}
\usepackage{listings}
\usepackage[noend]{algpseudocode}
\usepackage{algorithm}
\usepackage{enumitem}
\usepackage{framed}
\usepackage{relsize}
\usepackage{float}
\setmainfont[Mapping=TeX-text]{DejaVu Serif}
\newcommand{\tab}{\hspace*{3em}}

%\setmainfont{Arial}
%\setsansfont{FreeSans}
%\setmonofont{FreeMono}
\begin{document}
%\def\thesubsection {\alph{subsection}}
\renewcommand{\labelenumi}{\roman{enumi})}
\renewcommand{\labelenumii}{(\arabic{enumii})}

\begin{titlepage}
\begin{center}
\begin{figure}[t]
     \includegraphics[scale=0.7]{title/ntua_logo}
\end{figure}
\begin{LARGE}\textbf{ΕΘΝΙΚΟ ΜΕΤΣΟΒΙΟ ΠΟΛΥΤΕΧΝΕΙΟ\\}\end{LARGE}
\vspace{5cm}
\begin{Large}
ΣΧΟΛΗ ΗΜ\&ΜΥ\\
Αλγόριθμοι και Πολυπλοκότητα\\
1\textsuperscript{η} Σειρά Γραπτών Ασκήσεων\\
Ακ. έτος 2010-2011\\
\end{Large}
\vspace{18cm}
\begin{tabular}{l r}

\begin{flushright}
\Large{Γερακάρης Βασίλης}\\
\large{Α.Μ.: 03108092}\\
\end{flushright}
\end{tabular}\\
\vspace{1cm}

\vfill
\large\today\\
\end{center}
\end{titlepage}


%}}}

%{{{ Επιτροπή Αντιπροσώπων
\section{Επιτροπή Αντιπροσώπων (KT 4.15)} \setcounter{section}{1}
Ταξινομούμε τον πίνακα σε αύξουσα σειρά με βάση τα $f_i$. Βρίσκουμε το διάστημα (έστω m) με το§
μέγιστο $f_m$ που η αρχή του $s_m$ βρίσκεται πρίν το τέλος του 1ου, $f_1$ και
το επιλέγουμε ως αντιπρόσωπο.
Θα αποδείξουμε ότι η επιλογή που κάνουμε (σε κάθε βήμα) είναι και η βέλτιστη
δυνατή.

Έστω k, ένας διαφορετικός αντιπρόσωπος, που είναι η βέλτιστη επιλογή. Δεν
είναι δυνατόν να έχει δείκτη $k > m$, αφού έτσι δε θα επικάλυπτε το 1o διάστημα
(εξ'ορισμού του m). Αφού λοιπόν $k \leq m$ θα ισχύει και $f_k \leq f_m$. Άρα ο
k επικαλύπτει \textbf{το πολύ} όσα διαστήματα επικαλύπτει η επιλογή μας,
επομένως η επιμέρους λύση μας είναι η βέλτιστη.
Eπαγωγικά προκύπτει η oρθότητα του αλγορίθμου μας.

\begin{algorithm}[H]
\caption{Άσκηση 1}
\begin{algorithmic}[1]
    \State Sort A on ascending order using $f_i$ as key
\Procedure{RepresentantivesSelect}{$A, N$}
    \If{$N = 0$}
    \State \textbf{return} $RepresentativesList$
    \Else
    \State $i \gets 1$
    \While{$i \leq N$ \textbf{and} $s_i$ < $f_1$}
        \State $i \gets i+1$
    \EndWhile
    \State $i \gets i-1$
    \State $RepresentativesList.append (i)$
    \State $j \gets i+1$
    \While {$f_i$ > $s_j$}
	\State $A.remove(j)$
	\State $N \gets N-1$
	\State $j \gets j+1$
    \EndWhile
    \For{$j \gets i$ \textbf{to} 1 \textbf{step} -1}
	\State $A.remove (j)$
	\State $N \gets N-1$
    \EndFor
    \State $A.remove(i)$
    \State \textbf{return} RepresentativesSelect ($A, N$)
    \EndIf
\EndProcedure
\end{algorithmic}
\end{algorithm}

%}}}


%{{{ Βιαστικός Μοτοσυκλετιστής
\section{Βιαστικός Μοτοσυκλετιστής}
Ταξινομούμε τα διαστήματα σε αύξουσα σειρά με βάση τα όρια ταχυτήτων $v_i$ και
δημιουργούμε τον πίνακα χρόνων t, όπου $t_i = l_i/v_i$.
Έστω Τ ο χρόνος που θα ξεπεράσουμε το όριο ταχύτητας και u τα χιλιόμετρα κατά
τα οποία θα το ξεπεράσουμε. Έαν $T \leq t_1$, τότε κάνουμε Τ λεπτά με $v_1 +
u$ km/h, και τα υπόλοιπα με το όριο ταχύτητας, αλλιώς κάνουμε $t_1$ λεπτά
υπερβαίνοντας το όριο και επαναλαμβάνουμε τη διαδικασία για $T - t_1$ λεπτά
για το επόμενο διάστημα.

Θα αποδείξουμε ότι η επιλογή που κάνουμε σε κάθε βήμα, είναι η βέλτιστη
δυνατή.
Ο συνολικός χρόνος που χρειάζεται για να φτάσει στον προορισμό του είναι: \\
$T_{total}=\displaystyle\sum\limits_{i=1}^n{t_i} =
\displaystyle\sum\limits_{i=1}^n{\frac{l_i}{u_i}}$.

Η μέγιστη (ποσοστιαία) ελάττωση των χρονικών διαστημάτων προκύπτει αν
προσθέσουμε την ταχύτητα u στο μικρότερο παρονομαστή $v_i$. Kαθώς λοιπόν
όλες οι τιμές είναι δεδομένες και σταθερές ($T, v_i, l_i$), οποιοδήποτε άλλο
διάστημα αν επιλέγαμε, θα έδινε ποσοστιαίο κέρδος χρόνου \textbf{το πολύ} όσο αυτό
που υπολογίσαμε, επομένως η επιμέρους λύση μας είναι η βέλτιστη.\\

\begin{algorithm}[H]
\caption{Άσκηση 2}
\begin{algorithmic}[1]
    \State Sort A on ascending order using $v_i$ as key
    \For{$i \gets 1$ \textbf{to} N}
	\State $t_i \gets l_i / v_i$
    \EndFor
\Procedure{TimeSelect}{$A, N$}
    \State $i \gets 1$
    \While{$T \geq t_i$}
	\State $Selection.append (i,t_i)$
	\State $T \gets T - t_i$
        \State $i \gets i+1$
    \EndWhile
    \State $Selection.append (i, T)$
    \State \textbf{return} $Selection$
\EndProcedure
\end{algorithmic}
\end{algorithm}

Η πολυπλοκότητα του αλγορίθμου είναι $\Theta(n\log{n})$, όσο χρειάζεται η
ταξινόμηση. Στον ταξινομημένο πίνακα το αποτέλεσμα προκύπτει σε γραμμικό
χρόνο.
Στην περίπτωση που η υπέρβαση στο όριο γινόταν κατά παράγοντα α > 1, η επιλογή
του διαστήματος δε θα επηρρέαζε το αποτέλεσμα, αφού το ποσοστιαίο κέρδος θα
ήταν το ίδιο για όλα τα διαστήματα.

%}}}


%{{{ Βότσαλα στη Σκακιέρα
\section{Βότσαλα στη Σκακιέρα (DVP 6.5)}
\subsection{Σύγκριση με άπληστο αλγόριθμο}
Ο άπληστος αλγόριθμος μας εγγυάται ότι θα πάρει τουλάχιστον το 25\% της
βέλτιστης λύσης. Το χειρότερο πιθανό σενάριο για την άπληστη επιλογή είναι ένα
grid της παρακάτω μορφής, με N ένα μεγάλο θετικό αριθμό:
\begin{center}
    \begin{tabular}{| l | c | r | }
    \hline
    0 & Ν-1 & 0 \\ \hline
    Ν-1 & Ν & Ν-1 \\ \hline
    0 & Ν-1 & 0 \\ \hline
    0 & 0 & 0 \\
    \hline
    \end{tabular}
\end{center}

Στην περίπτωση αυτή ο λόγος της άπληστης λύσης προς τη βέλτιστη είναι
$Q = \frac{N}{4N-4}$. Tο όριο αυτής της ποσότητας όταν $N \to\infty$ είναι 1/4 ή
25\%.

\subsection{Βέλτιστη λύση με χρήση δυναμικού προγραμματισμού}
Έστω A,B,C,D οι 4 σειρές του προβλήματος. Μπορούμε να δημιουργήσουμε 8
διαφορετικούς έγκυρους συνδυασμούς επιλογής νομισμάτων, καθένας από τους
οποίους είναι συμβατός με ορισμένους, για την επιλογή στην επόμενη στήλη.
\begin{center}
    \begin{tabular}{ | c | c | c |}
    \hline
    i & k & k+1 \\ \hline \hline
    1 & $\emptyset$ & ALL \\ \hline
    2 & A & $\emptyset$,B,C,D,(B,D) \\ \hline
    3 & B & $\emptyset$,A,C,D,(A,C),(A,D) \\ \hline
    4 & C & $\emptyset$,A,B,D,(B,D),(A,D) \\ \hline
    5 & D & $\emptyset$,A,B,C,(A,C) \\ \hline
    6 & (A,C) & $\emptyset$,B,D,(B,D) \\ \hline
    7 & (B,D) & $\emptyset$,A,C,(A,C) \\ \hline
    8 & (A,D) & $\emptyset$,B,C \\
    \hline
    \end{tabular}
\end{center}

Θα λύσουμε το πρόβλημα με χρήση δυναμικού προγραμματισμού.


%}}}


%{{{ Χωρισμός Κειμένου σε Γραμμές
\section{Χωρισμός Κειμένου σε Γραμμές}
Test4
%}}}


%{{{ Αντίγραφα Αρχείου
\section{Αντίγραφα Αρχείου (KT 6.12)}
Test5
%}}}


%{{{Bonus: Έλεγχός Ταξινόμησης
\section{Bonus: Έλεγχος Ταξινόμησης}
Bonus!
%}}}

\end{document}
