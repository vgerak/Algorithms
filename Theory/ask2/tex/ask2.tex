%{{{Header
\documentclass[a4paper,11pt]{article}
\usepackage{anysize}
\marginsize{2cm}{2cm}{1cm}{1cm}
%\textwidth 6.0in \textheight = 664pt
\usepackage{xltxtra}
\usepackage{mathtools}
\usepackage{amsmath}
\usepackage{graphicx}
\usepackage{color}
\usepackage{xgreek}
\usepackage{fancyvrb}
\usepackage{minted}
\usepackage{listings}
\usepackage{hyperref}
\usepackage[noend]{algpseudocode}
\usepackage{algorithm}
\usepackage{enumitem}
\usepackage{framed}
\usepackage{relsize}
\usepackage{float}
\setmainfont[Mapping=TeX-text]{DejaVu Serif}
\newcommand{\tab}{\hspace*{3em}}

%\setmainfont{Arial}
%\setsansfont{FreeSans}
%\setmonofont{FreeMono}
\begin{document}
%\def\thesubsection {\alph{subsection}}
\renewcommand{\labelenumi}{\roman{enumi})}
\renewcommand{\labelenumii}{(\arabic{enumii})}

\begin{titlepage}
\begin{center}
\begin{figure}[t]
     \includegraphics[scale=0.7]{title/ntua_logo}
\end{figure}
\begin{LARGE}\textbf{ΕΘΝΙΚΟ ΜΕΤΣΟΒΙΟ ΠΟΛΥΤΕΧΝΕΙΟ\\}\end{LARGE}
\vspace{5cm}
\begin{Large}
ΣΧΟΛΗ ΗΜ\&ΜΥ\\
Αλγόριθμοι και Πολυπλοκότητα\\
1\textsuperscript{η} Σειρά Γραπτών Ασκήσεων\\
Ακ. έτος 2010-2011\\
\end{Large}
\vspace{18cm}
\begin{tabular}{l r}

\begin{flushright}
\Large{Γερακάρης Βασίλης}\\
\large{Α.Μ.: 03108092}\\
\end{flushright}
\end{tabular}\\
\vspace{1cm}

\vfill
\large\today\\
\end{center}
\end{titlepage}


%}}}

%{{{ Επιτροπή Αντιπροσώπων
\section{Επιτροπή Αντιπροσώπων (KT 4.15)} \setcounter{section}{1}
Ταξινομούμε τον πίνακα σε αύξουσα σειρά με βάση τα $f_i$. Βρίσκουμε το διάστημα (έστω m) με το
μέγιστο $f_m$ που η αρχή του $s_m$ βρίσκεται πρίν το τέλος του 1ου, $f_1$ και
το επιλέγουμε ως αντιπρόσωπο.
Θα αποδείξουμε ότι η επιλογή που κάνουμε (σε κάθε βήμα) είναι και η βέλτιστη
δυνατή.

Έστω k, ένας διαφορετικός αντιπρόσωπος, που είναι η βέλτιστη επιλογή. Δεν
είναι δυνατόν να έχει δείκτη $k > m$, αφού έτσι δε θα επικάλυπτε το 1o διάστημα
(εξ'ορισμού του m). Αφού λοιπόν $k \leq m$ θα ισχύει και $f_k \leq f_m$. Άρα ο
k επικαλύπτει \textbf{το πολύ} όσα διαστήματα επικαλύπτει η επιλογή μας,
επομένως η επιμέρους λύση μας είναι η βέλτιστη.
Eπαγωγικά προκύπτει η oρθότητα του αλγορίθμου μας.

\begin{algorithm}[H]
\caption{Άσκηση 1}
\begin{algorithmic}[1]
    \State Sort A on ascending order using $f_i$ as key
\Procedure{RepresentantivesSelect}{$A, N$}
    \If{$N = 0$}
    \State \textbf{return} $RepresentativesList$
    \Else
    \State $i \gets 1$
    \State $max \gets 0$
    \State $posmax \gets 0$
    \While{$i \leq N$}
        \If {$f_i > max$ \textbf{and} $s_i < f_1$}
	    \State $max \gets f_i$
	    \State $posmax \gets i$
	\EndIf
    \EndWhile
    \State $RepresentativesList.append (posmax)$
    \State $j \gets posmax$
    \While {$max$ > $s_j$}
	\State $A.remove(j)$
	\State $N \gets N-1$
	\State $j \gets j+1$
    \EndWhile
    \For{$j \gets posmax$ \textbf{to} 1 \textbf{step} -1}
	\State $A.remove (j)$
	\State $N \gets N-1$
    \EndFor
    \State \textbf{return} RepresentativesSelect ($A, N$)
    \EndIf
\EndProcedure
\end{algorithmic}
\end{algorithm}

%}}}

%{{{ Βιαστικός Μοτοσυκλετιστής
\section{Βιαστικός Μοτοσυκλετιστής}
Ταξινομούμε τα διαστήματα σε αύξουσα σειρά με βάση τα όρια ταχυτήτων $v_i$ και
δημιουργούμε τον πίνακα χρόνων t, όπου $t_i = l_i/v_i$.
Έστω Τ ο χρόνος που θα ξεπεράσουμε το όριο ταχύτητας και u τα χιλιόμετρα κατά
τα οποία θα το ξεπεράσουμε. Έαν $T \leq t_1$, τότε κάνουμε Τ λεπτά με $v_1 +
u$ km/h, και τα υπόλοιπα με το όριο ταχύτητας, αλλιώς κάνουμε $t_1$ λεπτά
υπερβαίνοντας το όριο και επαναλαμβάνουμε τη διαδικασία για $T - t_1$ λεπτά
για το επόμενο διάστημα.

Θα αποδείξουμε ότι η επιλογή που κάνουμε σε κάθε βήμα, είναι η βέλτιστη
δυνατή.
Ο συνολικός χρόνος που χρειάζεται για να φτάσει στον προορισμό του είναι: \\
$T_{total}=\displaystyle\sum\limits_{i=1}^n{t_i} =
\displaystyle\sum\limits_{i=1}^n{\frac{l_i}{u_i}}$.

Η μέγιστη (ποσοστιαία) ελάττωση των χρονικών διαστημάτων προκύπτει αν
προσθέσουμε την ταχύτητα u στο μικρότερο παρονομαστή $v_i$. Kαθώς λοιπόν
όλες οι τιμές είναι δεδομένες και σταθερές ($T, v_i, l_i$), οποιοδήποτε άλλο
διάστημα αν επιλέγαμε, θα έδινε ποσοστιαίο κέρδος χρόνου \textbf{το πολύ} όσο αυτό
που υπολογίσαμε, επομένως η επιμέρους λύση μας είναι η βέλτιστη.\\

\begin{algorithm}[H]
\caption{Άσκηση 2}
\begin{algorithmic}[1]
    \State Sort A on ascending order using $v_i$ as key
    \For{$i \gets 1$ \textbf{to} N}
	\State $t_i \gets l_i / v_i$
    \EndFor
\Procedure{TimeSelect}{$A, N$}
    \State $i \gets 1$
    \While{$T \geq t_i$}
	\State $Selection.append (i,t_i)$
	\State $T \gets T - t_i$
        \State $i \gets i+1$
    \EndWhile
    \State $Selection.append (i, T)$
    \State \textbf{return} $Selection$
\EndProcedure
\end{algorithmic}
\end{algorithm}

Η πολυπλοκότητα του αλγορίθμου είναι $\Theta(n\log{n})$, όσο χρειάζεται η
ταξινόμηση. Στον ταξινομημένο πίνακα το αποτέλεσμα προκύπτει σε γραμμικό
χρόνο.
Στην περίπτωση που η υπέρβαση στο όριο γινόταν κατά παράγοντα α > 1, η επιλογή
του διαστήματος δε θα επηρρέαζε το αποτέλεσμα, αφού το ποσοστιαίο κέρδος θα
ήταν το ίδιο για όλα τα διαστήματα.

%}}}

%{{{ Βότσαλα στη Σκακιέρα
\section{Βότσαλα στη Σκακιέρα (DVP 6.5)}
\subsection{Σύγκριση με άπληστο αλγόριθμο}
Ο άπληστος αλγόριθμος μας εγγυάται ότι θα πάρει τουλάχιστον το 25\% της
βέλτιστης λύσης. Το χειρότερο πιθανό σενάριο για την άπληστη επιλογή είναι ένα
grid της παρακάτω μορφής, με N ένα μεγάλο θετικό αριθμό:
\begin{center}
    \begin{tabular}{| c | c | c | }
    \hline
    0 & Ν-1 & 0 \\ \hline
    Ν-1 & Ν & Ν-1 \\ \hline
    0 & Ν-1 & 0 \\ \hline
    0 & 0 & 0 \\
    \hline
    \end{tabular}
\end{center}

Στην περίπτωση αυτή ο λόγος της άπληστης λύσης προς τη βέλτιστη είναι
$Q = \frac{N}{4N-4}$. Tο όριο αυτής της ποσότητας όταν $N \to\infty$ είναι 1/4 ή
25\%.

\subsection{Βέλτιστη λύση με χρήση δυναμικού προγραμματισμού}
Έστω A,B,C,D οι 4 σειρές του προβλήματος. Μπορούμε να δημιουργήσουμε 8
διαφορετικούς έγκυρους συνδυασμούς επιλογής νομισμάτων, καθένας από τους
οποίους είναι συμβατός με ορισμένους, από την επιλογή της προηγούμενης στήλης.
\begin{center}
    \begin{tabular}{ | c | c | c | c | c |}
    \hline
    i & Selection & Profit(k) & Valid Prev. Status & Valid(k-1) \\ \hline \hline
    0 & $\emptyset$ & 0 & ALL & {0,1,2,3,4,5,6,7}\\ \hline
    1 & A & Val[A] & $\emptyset$,B,C,D,(B,D) & {0,2,3,4,6}\\ \hline
    2 & B & Val[B] & $\emptyset$,A,C,D,(A,C),(A,D) & {0,1,3,4,5,7}\\ \hline
    3 & C & Val[C] & $\emptyset$,A,B,D,(B,D),(A,D) & {0,1,2,4,6,7}\\ \hline
    4 & D & Val[D] & $\emptyset$,A,B,C,(A,C) & {0,1,2,3,4}\\ \hline
    5 & (A,C) & Val[A] + Val[C] & $\emptyset$,B,D,(B,D) & {0,2,4,6}\\ \hline
    6 & (B,D) & Val[B] + Val[D] & $\emptyset$,A,C,(A,C) & {0,1,3,5}\\ \hline
    7 & (A,D) & Val[A] + Val[D] & $\emptyset$,B,C & {0,2,3}\\
    \hline
    \end{tabular}
\end{center}

Θα λύσουμε το πρόβλημα με χρήση δυναμικού προγραμματισμού.
Θεωρούμε τον πίνακα $P[0..N][0..7]$, όπου κάθε γραμμή αντιστοιχεί στον αντίστοιχο
συνδυασμό που καταγράφεται στον παραπάνω πίνακα. Δεδομένης της σκακιέρας
Val (4 x N) με τις αντίστοιχες τιμές, προκύπτει ο παρακάτω αλγόριθμος:

\begin{algorithm}[H]
\caption{Άσκηση 3}
\begin{algorithmic}[1]
\Procedure{OptimalPebbling}{$Val, N$}
    \For{$i \gets 0$ \textbf{to} 7}
	\State $P[0][i] \gets 0$
    \EndFor
    \For{$i \gets 1$ \textbf{to} N}
	\For{$j \gets 0$ \textbf{to} 7}
	    \State $P[i][j] = Profit[i] + max\{ \forall k \in Valid(j-1),P[i-1][k]\}$
	\EndFor
    \EndFor
    \State $OPT = 0$
    \For{$j \gets 0$ \textbf{to} 7}
	\If {$OPT < P[N][i]$}
	    \State $OPT \gets P[N][i]$
	\EndIf
    \EndFor
    \State \textbf{return} $OPT$
\EndProcedure
\end{algorithmic}
\end{algorithm}

Υπάρχουν 41 διαφορετικοί συνδυασμοί σε κάθε βήμα, οπότε η χρονική
πολυπλοκότητα του αλγορίθμου είναι: $\Theta{(41*N)} = \Theta{(N)}$

%}}}

%{{{ Χωρισμός Κειμένου σε Γραμμές
\section{Χωρισμός Κειμένου σε Γραμμές}
Θεωρούμε τη συνάρτηση κόστους γραμμής που περιέχει τις λέξεις από i εώς και j
ως \begin{center}$v(i,j) = s_k^2 = (C + 1 - \sum\limits_{p = 1}^{j} (l_p
+1))^2$ \end{center}
Στην περίπτωση που η παραπάνω τιμή είναι αρνητική (δηλαδή η λέξη δε χωράει στη
γραμμή), η v επιστρέφει $\infty$.\\
Ορίζουμε την αναδρομική σχέση:
OPT(j) =$
\begin{dcases}
    v(1,j) & \text{if } c(1,j) < \infty\\
    \min\limits_{1 \leq k < j} (f(k) + v(k + 1, j)) & \text{if } c(1,j) =
    \infty
\end{dcases}$

Και με χρήση δυναμικού προγραμματισμού προκύπτει η λύση σε χρονο O($j^2$).

Αξίζει να σημειωθεί ότι αν κάναμε χρήση των ευρημάτων των Galil, Zvi, Park και
Kunsoo
(όπως δημοσιεύτηκαν στο paper \href{http://www.cs.ust.hk/mjg_lib/bibs/DPSu/DPSu.Files/GaPa90.PDF}{A
linear-time algorithm for concave one-dimensional dynamic programming}),
εκμεταλλευόμενοι τη φυσιολογία του προβήματος, θα
μπορούσαμε να κατεβάσουμε την πολυπλοκότητα σε γραμμικό χρόνο.

%}}}

%{{{ Αντίγραφα Αρχείου
\newpage
\section{Αντίγραφα Αρχείου (KT 6.12)}
Έστω OPT(j) το ελάχιστο κόστος μιας διευθέτησης που καλύπτει τους servers 1
εώς j, θεωρώντας ότ τοποθετούμε ένα αντίγραφο στον server j. Ψάχνουμε τις
πιθανές θέσεις (i) με i < j που μπορούμε να τοποθετήσουμε ένα αντίγραφο
καταλήγωντας σε βέλτιστη λύση. Έστω ότι βρίσκεται στη θέση i = k.
Το συνολικό κόστος που προκύπτει για το OPT(j) είναι ίσο με το άθροισμα:
\begin{list}{*}{}
\item Του κόστους $c_j$
\item Του OPT(k), αφού ώς το k θεωρούμε ότι επιλέξαμε βέλτιστα
\item Του κόστους πρόσβασης για κάθε server από $i \to j$, το οποίο ισούται με
\begin{center} $\sum\limits_{i = k}^{j} b_k (j-k)$ \end{center}
\end{list}
Από τις j (σε πλήθος) αυτές λύσεις, επιλέγουμε τη βέλτιστη, αυτή με το
ελάχιστο κόστος. Επομένως η αναδρομική μας σχέση είναι η εξής:
\begin{center}
$OPT(j) = c_j + \min\limits_{0 \leq i \leq j}(OPT(i)+\sum\limits_{i = k}^{j}
b_k (j-k))$
\end{center}

Θεωρώντας ότι OPT(0) = 0 μπορούμε να παράγουμε τις τιμές του OPT αυξάνοντας το
j. Η βέλτιστη λύση έχει τιμή OPT(n) και για να βρούμε τη διευθέτηση που την
προκάλεσε κάνουμε ένα backtracking, διατηρώντας τα i από τα OPT(i) στοιχεία
που επιλέγονταν.

Για τον υπολογισμό του κάθε j χρησιμοποιούμε O(j) χρόνο, άρα η τελική
πολυπλοκότητα είναι $O(n^2)$.
%}}}

\end{document}
