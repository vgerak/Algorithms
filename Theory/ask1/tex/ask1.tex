%{{{Header
\documentclass[a4paper,10pt]{article} \usepackage{anysize}
\marginsize{2cm}{2cm}{1cm}{1cm}
%\textwidth 6.0in \textheight = 664pt
\usepackage{xltxtra}
\usepackage{graphicx}
\usepackage{color}
\usepackage{xgreek}
\usepackage{fancyvrb}
\usepackage{minted}
\usepackage{listings}
\usepackage{enumitem}
\usepackage{framed}
\usepackage{relsize}
\usepackage{float}
\setmainfont[Mapping=TeX-text]{DejaVu Serif}
\newcommand{\tab}{\hspace*{3em}}

%\setmainfont{Arial}
%\setsansfont{FreeSans}
%\setmonofont{FreeMono}
\begin{document}
%\def\thesubsection {\alph{subsection}}
\renewcommand{\labelenumi}{\roman{enumi})}
\renewcommand{\labelenumii}{(\arabic{enumii})}

\begin{titlepage}
\begin{center}
\begin{figure}[t]
     \includegraphics[scale=0.7]{title/ntua_logo}
\end{figure}
\begin{LARGE}\textbf{ΕΘΝΙΚΟ ΜΕΤΣΟΒΙΟ ΠΟΛΥΤΕΧΝΕΙΟ\\}\end{LARGE}
\vspace{5cm}
\begin{Large}
ΣΧΟΛΗ ΗΜ\&ΜΥ\\
Αλγόριθμοι και Πολυπλοκότητα\\
1\textsuperscript{η} Σειρά Γραπτών Ασκήσεων\\
Ακ. έτος 2010-2011\\
\end{Large}
\vspace{18cm}
\begin{tabular}{l r}

\begin{flushright}
\Large{Γερακάρης Βασίλης}\\
\large{Α.Μ.: 03108092}\\
\end{flushright}
\end{tabular}\\
\vspace{1cm}

\vfill
\large\today\\
\end{center}
\end{titlepage}


%}}}

%{{{Ταξινόμηση
\section{Ασυμπτωτικός συμβολισμός, Αναδρομικές
Σχέσεις} \setcounter{section}{1}
\begin{enumerate}

\item Ταξινόμηση
\begin{enumerate}
\item $\log {n^3} = 3 \log{n} = \Theta(logn)$
\item $\sqrt{n}*\log^{50}{n} $
\item $\frac{n}{\log{\log{n}} } $
\item $\log{n!} = \Theta (nlog{n})$
\item $n*\log^{10}{n} $
\item $n^{1.01} $
\item $5^{\log_{2}{n}} = n^{log_{2}{5}} = \Theta (n^{2.322}) $
\item $\sum_{k=1}^n{k^5} = {k^6}$
\item $\log^{\log{n}}{n} = n^{\log{\log{n}} } $
\item $2^{\log_2^4{n}} $
\item $\log^{\sqrt{n}}{n} $
\item $e^{\frac{n}{\ln{n}} } $
\item $n3^n $
\item $2^{2n} $
\item $\sqrt{n!} $
\end{enumerate}
%}}}

%{{{Τάξη μεγέθους
\item Τάξη Μεγέθους
\begin{enumerate}
\item $T(n)=5T(n/7)+n\log{n}$ ,\tab ($a=5 , b=7$) \\
$\Rightarrow n^{\log _7 {5}} = n^{0.827}$
$\Rightarrow n^{0.827} < n < n\log{n}$\\
$\Rightarrow T(n) \in \Theta(n\log{n})$ \tab M.Th (case \#3)\\


\item $T(n) = 4T(n/5)+n/\log^2{n}$ ,\tab ($a=4 , b=5$)\\
$\Rightarrow n^{\log_5 {4}} = n^{0.861}$
$\Rightarrow n^{0.861} < n/\log^2{n}$\\
$\Rightarrow T(n) \in \Theta(n/\log^2{n})$ \tab M.Th (case \#3)\\

\item $T(n) = T(n/3) +3T(n/7)+n$
\tab ($\frac{1}{3}+\frac{1}{7}+\frac{1}{7}+\frac{1}{7}< 1)$\\
$\Rightarrow T(n) \in \Theta(n)$ \tab M.Th (special case) \\

\item $T(n) = 6T(n/6)+n$\\
$\Rightarrow n^{\log _6 {6}} = n$\\
$\Rightarrow T(n) \in \Theta(n\log {n})$ \tab M.Th (sp. case \#2)\\

\item $T(n) = T(n/3) + T(2n/3) + n$\\
$\Rightarrow T(n) \in \Theta(n\log{n})$\\

\item $T(n) = 16T(n/4) + n^3\log^2{n}$ , \tab ($a=16, b=4)$\\
$\Rightarrow n^{\log _4{16}} = n^2$
$\Rightarrow n^2 < n^3\log^2{n}$\\
$\Rightarrow T(n) \in \Theta(n^3\log^2{n})$ \tab M.Th (case \#3)\\

\item $T(n) = T(\sqrt{n}) + \Theta(\log{\log{n}})$ \tab
(Θέτω $k=\log{n}, f(k)=T(n))$\\
$\Rightarrow
T(\sqrt{n})=f(\sqrt{k})=f(\log{\sqrt{n}})=f(\frac{\log{n}}{2})=f(\frac{k}{2}) $\\
$\Rightarrow f(k)=f(\frac{k}{2})+\Theta(\log{k})$\\
$\Rightarrow \Theta(n)$ \tab (M.Th (case \#1)\\

\item $T(n) = T(n-3) + \log{n}$ \tab ($\frac{n}{3} * \log{n})$ \\
$\Rightarrow T(n) \in \Theta(nlogn)$

\end{enumerate}
\end{enumerate}
\pagebreak
%}}}

%{{{Ταξινόμηση σε Πίνακα με Πολλά Ίδια Στοιχεία
\section{Ταξινόμηση σε Πίνακα με Πολλά Ίδια Στοιχεία}




Για να κάνουμε ταξινόμηση σε ένα τέτοιο πίνακα σε πρώτο στάδιο μετράμε τα
στοιχεία που είναι ίδια σε κάθε πέρασμα και τα βγάζουμε από τον πίνακα. Για να
ολοκληρώσουμε αυτή τη διαδικασία και με δεδομένο πως το πλήθος των
διαφορετικών στοιχείων είναι μόλις $\log^d{n}$ προκύπτει ο αναδρομικός τύπος
$T(n)=T(n*\frac{m-1}{m})+f(n),f(n) \in O(n),m=\log^d{n}$. Επιλύοντας με χρήση
του \textit{Master Theorem} καταλήγουμε σε πολυπλοκότητα $O(n)$ για το πρώτο
βήμα. Έπειτα με mergesort ταξινομούμε τα στοιχεία που έχουμε και τα
επεκτείνουμε ανάλογα με το πλήθος του καθενός έτσι ώστε το τελικό αποτέλεσμα
να είναι και πάλι μήκους $n$. Η διαδικασία ολοκληρώνεται σε άλλο
$\log^d{n}*\log{\log^d{n}}=d*\log^d{n}*\log{\log{n}}$. Προφανώς το κύριο
κομμάτι του χρόνου  καταναλώνεται στο δεύτερο στάδιο. Συνεπώς η πολυπλοκότητα
του αλγορίθμου είναι $\Theta(\log^d{n}*\log{\log{n}})$ άρα και
$O(n*\log{\log{n}})$.

%}}}

%{{{Δυαδική Αναζήτηση
\section{Δυαδική Αναζήτηση}
\begin{enumerate}
\item Ξεκινάμε (για $i=1$) συγκρίνοντας το $A[i]$ με το $x$. Όσο το $x$ είναι
μεγαλύτερος από αυτό, κρατάμε την τιμή του $(i+1)$ σε μια άλλη μεταβλητή $k$ και
διπλασιάζουμε το i.
Μόλις φτάσουμε σε μεγαλύτερο αριθμό από τον $x$, εφαρμόζω ένα έλεγχο.
Αν το στοιχείο $A[i]$ είναι διάφορο του $\infty$, τότε εφαρμόζω δυαδική αναζήτηση
για το στοιχείο $x$ στο τμήμα $Α[k..i]$ του πίνακα και έχω το αποτέλεσμα για
πιθανή ύπαρξη \& θέση του στοιχείου. \\
Αν το στοιχείο $A[i]$ είναι ίσο με $\infty$, τότε ελέγχω αν $k=i$ (τερματική
συνθήκη) και αν όχι ελέγχω το $A[(k+i) div 2]$.
\begin{itemize}
\item Αν είναι $\infty$ τότε θέτω $i= (k+i)div 2$ και επαναλαβάνω.
\item Αν είναι αριθμός $m$ τον συγκρίνω με το $x$.
\begin{itemize}
\item Αν $x>m$ τότε θέτω $k= (k+i) div 2$ και επαναλαμβάνω.
\item Αν $x<m$ τότε εφαρμόζω δυαδική αναζήτηση στο τμήμα $A[k..i]$ του
πίνακα.
\item Αν $x=m$ τότε είναι το ζητούμενο στοιχείο.
\end{itemize}
\end{itemize}

Η τελική πολυπλοκότητα είναι $O(\log{n})$.\\

\item Εργαζόμαστε με τα k πρώτα στοιχεία από κάθε πίνακα (συμπληρώνοντας τις
κενές θέσεις με + $\infty$ , αν ο πίνακας δεν έχει τόσα στοιχεία). Συγκρίνουμε τα
$A[k/2]$ με $B[k/2+1]$ και $A[k/2+1]$ με $B[k/2]$.
\begin{itemize}
\item Αν $A[k/2]<B[k/2+1]$ και $B[k/2]<A[k/2+1]$ τοτε το k-οστό στοιχείο είναι το
$max\{A[k/2],B[k/2]\}$.
\item Αν $A[k/2]<B[k/2+1]$ και $B[k/2]>A[k/2+1]$ τότε επαναλαμβάνουμε αναδρομικά
τη διαδικασία στους υποπίνακες $A[k/2+1]..A[k]$ και $B[1]..B[k/2]$.
\end{itemize}
\end{enumerate}
%}}}

%{{{Συλλογή Comics
\section{Συλλογή Comics}
Ελέγχουμε αρχικά το 1\textsuperscript{o} bit για όλα τα τεύχη, χωρίζοντάς τα
σε 2 σύνολα με πλήθος στοιχείων $\frac{n}{2}$ και $\frac{n}{2}-1$. Το τέυχος
που λείπει είναι στο σύνολο με τα λιγότερα στοιχεία, οπότε αποθηκεύουμε την
τιμή αυτού του bit. \\
Εφαρμόζουμε αναδρομικά τη μέθοδο για το επόμενο bit, κάθε φορά στο σύνολο με τα λιγότερα στοιχεία,
κάνοντας τις μισές ερωτήσεις από το προηγούμενο βήμα σε κάθε αναδρομή. \\
Τελικά, θα καταλήξουμε στον κωδικό του τεύχους που λείπει από τη συλλογή, έχοντας
ρωτήσει $n+\frac{n}{2}+\frac{n}{4}+..+1 \approx 2n$ φορές.\\
Πρακτικά, είναι: $(n-1)+(\frac{n}{2}-1)+(\frac{n}{4} -1)+..=
\displaystyle\sum\limits_{i=0}^{\log{n}} (\frac{n}{2^i} -1) = 2n - log{n} -2$


\pagebreak
%}}}

%{{{Πολυκατοικίες χωρίς Θέα
\section{Πολυκατοικίες χωρίς Θέα}
Αρχίζουμε αρχικοποιώντας $B[1] = 0$, μιάς και ο 1\textsuperscript{ος} δεν έχει
κτήριο δυτικά του.\\
Για κάθε κτήριο i ανατολικά του, θέτουμε $k= i-1$ και συγκρίνουμε το ύψος του $A[i]$ με του
$A[k]$.
\begin{itemize}
\item Αν $A[i]<A[k]$, τότε $B[i] = k$ και συνέχιζουμε στο επόμενο δεξιά
κτήριο ($i= i+1$) .
\item Αν $A[i] \ge A[k]$,
\begin{itemize}
\item Αν $B[k] = 0$ τότε $B[i] = 0$ και συνέχιζουμε στο επόμενο δεξιά
κτήριο.
\item Αλλιώς θέτουμε $k = B[k]$ και επαναλαμβάνουμε τον έλεγχο.
\end{itemize}
\end{itemize}

%}}}

\end{document}
